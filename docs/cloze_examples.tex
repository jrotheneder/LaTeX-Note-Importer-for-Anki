\documentclass[12pt]{article}
\input{../preambles/preview_preamble.tex}
\begin{document}

\section{Testing Clozes}
\tags{mathematik::TestingAnki}

% Some examples taken from https://www.reddit.com/r/Anki/comments/y99mqk/how_i_use_anki_for_mathematics/
\begin{note}
    \xplain{sujg-47ky-fcvp}
    \newline
    \begin{clozefield}
        A binary operation $ · $ on a set $ M $ is called 
        \begin{cloze}
            commutative,
        \end{cloze}
        if 
        \begin{cloze} $ a · b = b ·a  $ for all $ a,b ∈ M $. \end{cloze}
    \end{clozefield} 
    \begin{field} \end{field}
\end{note}

\begin{note}
    \xplain{kjbu-lex0-7lvj}
    \begin{clozefield}
        Let $ X,Y $ be sets. The notation \begin{cloze} $ f\colon X → Y $ \end{cloze}
        means that \begin{cloze} $ f $ is a function from $ X $ to $ Y $. \end{cloze}
    \end{clozefield}
\end{note}

\begin{note}
    \xplain{xoaj-bll2-ndol}
    \begin{clozefield}
        The equation 
        \begin{cloze}
            \begin{equation*}
                { (x-a)^{2}+(y-b)^{2}=r^{2}.}
            \end{equation*}
        \end{cloze}
        describes a \begin{cloze} circle \end{cloze}
        with coordinates \begin{cloze}  $(a,b)$ \end{cloze} 
        and radius \begin{cloze} $ r $ \end{cloze}.
    \end{clozefield}
\end{note}



\end{document}


