\documentclass[12pt]{article}
\special{papersize=3in,5in}
\usepackage[utf8]{inputenc}
\usepackage{amssymb,amsmath}
\pagestyle{empty}
\setlength{\parindent}{0in}
\newcommand{\detail}[1]{{\scriptsize(#1)\par}~}
\newcommand{\refs}[1]{{\scriptsize\textit{Refs: }#1\par}\hfill.}
\newcommand*{\abs}[1]{\left\vert#1\right\vert}
\newcommand*{\NN}{\mathbb{N}}
\newcommand*{\ZZ}{\mathbb{Z}}
\newcommand*{\QQ}{\mathbb{Q}}
\newcommand*{\FF}{\mathbb{F}}
\newcommand*{\PP}{\mathbb{P}}
\newcommand*{\units}{\times}

\usepackage{xcolor} 

%%% commands that do not need to imported into Anki:
\usepackage{mdframed}
\newcommand*{\tags}[1]{\paragraph{tags: }#1\bigskip}
\newcommand*{\xfield}[1]{\begin{mdframed}\centering #1\end{mdframed}\bigskip}
\newenvironment{field}{}{}
\newcommand*{\xplain}[1]{\begin{mdframed}\texttt{#1}\end{mdframed}\bigskip}
\newenvironment{plain}{\ttfamily}{\par}
\newenvironment{note}{}{}
\newenvironment{clozefield}{}{}
\newenvironment{cloze}{\textcolor{red}{$\{$}}{\textcolor{red}{$\}$}}
% END OF THE PREAMBLE
\begin{document}

\section{Testing Clozes}
\tags{mathematik::TestingAnki}

% Some examples taken from https://www.reddit.com/r/Anki/comments/y99mqk/how_i_use_anki_for_mathematics/
\begin{note}
    \xplain{sujg-47ky-fcvp}
    \begin{clozefield}
        A binary operation $ \cdot $ on a set $ M $ is called 
        \begin{cloze}
            commutative,
        \end{cloze}
        if 
        \begin{cloze} $ a \cdot b = b \cdot a  $ for all $ a,b \in M $. \end{cloze}
    \end{clozefield} 
    \begin{field} \end{field}
\end{note}

\begin{note}
    \xplain{kjbu-lex0-7lvj}
    \begin{clozefield}
        Let $ X,Y $ be sets. The notation \begin{cloze} $ f\colon X \rightarrow
        Y $ \end{cloze}
        means that \begin{cloze} $ f $ is a function from $ X $ to $ Y $. \end{cloze}
    \end{clozefield}
\end{note}

\begin{note}
    \xplain{xoaj-bll2-ndol}
    \begin{clozefield}
        The equation 
        \begin{cloze}
            \begin{equation*}
                { (x-a)^{2}+(y-b)^{2}=r^{2}.}
            \end{equation*}
        \end{cloze}
        describes a 
        \begin{cloze} circle \end{cloze} centered at \begin{cloze}  
        $(a,b)$ \end{cloze} with radius \begin{cloze} 
        $ r $ \end{cloze}.
    \end{clozefield}
\end{note}



\end{document}


